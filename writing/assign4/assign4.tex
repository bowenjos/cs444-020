\documentclass[letterpaper,10pt,titlepage]{article}

\usepackage{graphicx}                                        
\usepackage{amssymb}                                         
\usepackage{amsmath}                                         
\usepackage{amsthm}                                          

\usepackage{alltt}                                           
\usepackage{float}
\usepackage{color}
\usepackage{url}

\usepackage{balance}
\usepackage[TABBOTCAP, tight]{subfigure}
\usepackage{enumitem}
\usepackage{pstricks, pst-node}

\usepackage{geometry}
\usepackage{listings}  
\lstset{language=C}
\geometry{textheight=8.5in, textwidth=6in}

%random comment

\newcommand{\cred}[1]{{\color{red}#1}}
\newcommand{\cblue}[1]{{\color{blue}#1}}

\usepackage{hyperref}
\usepackage{geometry}

\def\name{Christopher Mendez, Joshua Bowen}


%% The following metadata will show up in the PDF properties
\hypersetup{
  colorlinks = true,
  urlcolor = black,
  pdfauthor = {\name},
  pdfkeywords = {cs444 },
  pdftitle = {CS 444 Project 4: The SLOB SLAB},
  pdfsubject = {CS 444 Project 4},
  pdfpagemode = UseNone
}

\begin{document}

\begin{titlepage}
	\centering
	\vspace*{4cm}
	{\scshape\huge Assignment 4: The SLOB SLAB\par}
	\vspace{1cm}
	{\scshape\LARGE CS 444: Operating Systems II\par}
	\vspace{0.5cm}
	{\large\bfseries Fall 2016\par}
	{\large Abstract\par}
	\vspace {0.5cm}
		This project involves researching and implementing a SLAB memory manager with best fit algorithms. The goal is to gain an understanding of how Memory Management works inside of an operating system. Through work our process of learning about and implementing the SLAB is shown.
	\par
	\vspace{1cm}
	{\Large\itshape Joshua D. Bowen\par}
	{\Large\itshape Chris J. Mendez\par}
	\vfill
	{\large \today\par}	

\end{titlepage}

Assignment 4, CS444
Christopher Mendez, Joshua Bowen



\section{MM Driver Design}

Our first decision was to research the already existing SLOB.
We found in the SLOB the already existent Memory Allocation system and then the memory placement portion.
The SLOB already had a system in place to place or create pages at the first location that it could find space.
The modification we needed to make is one that takes this but instead of placing at the first spot, it instead finds the best spot.

To implement this design we first came up with a mental model of what this looked like.

The SLOB would iterate accross the list of pages looking for the best fit.
When the current position is a better fit than the current best fit, the new best fit is remembered.
Once the SLOB reaches the end of the list it will place the item at the bes fit location.
If no fit is found at all, then a new page is generated to fit the item.

Using this mental model we knew what it was that we needed to do.
With this in mind we sketched up some pseudocode.




\section{Questions}
\subsection{What do you think the main point of this assignment is?}

 \sloppy The main point of this assignment was to learn more about the various aspects within the kernel of the Operating Systems.
 In this specific assignment we were supposed to be able to look at the SLOB and find what it was that needed to be changed.
 Then as in previous assignments we were responsible for getting it working with the rest of the OS.
 We continued with the tradition of having some strange error or bug in the code that we needed to flex our problem solving skills to solve.

\sloppy On top of the figuring out how to get the SLAB working, we were also tasked with how to test it.
I think one of the points of this assignment was to once again make sure that we were able to understand our assignment and make tests that thoroughly tested the code.
In general these assignments often have different ways of testing the algorithm that is different from every other part of the kernel.

\subsection{How did you personally approach the problem? Design decisions, algorithm, etc.}

\sloppy The first step, of course, was to research the already existing SLOB. 
After looking into the SLOB we were able to determine what would need to be changed. 
In general the SLOB already had a functioning Memory Allocation system which didn't need to be changed. 
What we needed to change was how how the SLOB determined where pages went. Our goal is to change the "First Find" fit to a "Best Fit".

\sloppy We decided that all we needed to change was this aspect of the SLOB. 
We then wrote up a design for the algorithm before implementing.
This design can be seen above.

\sloppy After crafting our design we began implementation.
After implementation came testing.


\subsection{How did you ensure your solution was correct? Testing details, for instance.}

\sloppy  

\subsection{What did you learn?}

\sloppy  We learned about the SLOB and memory management in general.
We also learned how to debug and test a different aspect of the kernel.
Additionally, we learned how to familiarze ourselves with another aspect of the kernel and make changes to what we needed so that it would do what we wanted.


\section {History}
\subsection {Git Version Control Log}
\begin{tabular}{ |p{10cm}|p{3cm}| } 
 \hline
 Commit & Message \\
 \hline


\hline

\end{tabular}

\subsection{Work Log}
\begin{tabular}{ |p{3cm}|p{3cm}|p{5cm}| } 
\hline
Date & Time & What \\
\hline
11/15/2016& 3pm-5m& Began researching the currently implemented SLOB and gaining and understanding of how the memory management works. \\
 \hline
 11/19/2016& 10am-12pm& Began some code. \\
 \hline
 11/20/2016& 10am- & Continued working on the assignment, began writing our formal design document.\\
\end{tabular}

\end{document}
